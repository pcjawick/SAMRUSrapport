% !TEX TS-program = pdflatex
% !TEX encoding = UTF-8 Unicode

% This is a simple template for a LaTeX document using the "article" class.
% See "book", "report", "letter" for other types of document.

\documentclass[11pt]{report} % use larger type; default would be 10pt

\usepackage[utf8]{inputenc} % set input encoding (not needed with XeLaTeX)

%%% Examples of Article customizations
% These packages are optional, depending whether you want the features they provide.
% See the LaTeX Companion or other references for full information.

%%% PAGE DIMENSIONS
\usepackage{geometry} % to change the page dimensions
\geometry{a4paper} % or letterpaper (US) or a5paper or....
% \geometry{margin=2in} % for example, change the margins to 2 inches all round
% \geometry{landscape} % set up the page for landscape
%   read geometry.pdf for detailed page layout information

\usepackage{graphicx} % support the \includegraphics command and options

% \usepackage[parfill]{parskip} % Activate to begin paragraphs with an empty line rather than an indent

%%% PACKAGES
\usepackage{booktabs} % for much better looking tables
\usepackage{array} % for better arrays (eg matrices) in maths
\usepackage{paralist} % very flexible & customisable lists (eg. enumerate/itemize, etc.)
\usepackage{verbatim} % adds environment for commenting out blocks of text & for better verbatim
\usepackage{subfig} % make it possible to include more than one captioned figure/table in a single float
\usepackage{titlesec}%Følgende tre lagt til for kryssref og forside
\usepackage{afterpage}
\usepackage{hyperref}
\usepackage{xcolor}

%%% HEADERS & FOOTERS
\usepackage{fancyhdr} % This should be set AFTER setting up the page geometry
\pagestyle{fancy} % options: empty , plain , fancy
\renewcommand{\headrulewidth}{0pt} % customise the layout...
\lhead{}\chead{}\rhead{}
\lfoot{}\cfoot{\thepage}\rfoot{}

%%% SECTION TITLE APPEARANCE
\usepackage{sectsty}
\allsectionsfont{\sffamily\mdseries\upshape} % (See the fntguide.pdf for font help)
% (This matches ConTeXt defaults)

%%% ToC (table of contents) APPEARANCE
\usepackage[nottoc,notlof,notlot]{tocbibind} % Put the bibliography in the ToC
\usepackage[titles,subfigure]{tocloft} % Alter the style of the Table of Contents
\renewcommand{\cftsecfont}{\rmfamily\mdseries\upshape}
\renewcommand{\cftsecpagefont}{\rmfamily\mdseries\upshape} % No bold!

%%% END Article customizations

%%% The "real" document content comes below...

\title{Nedre Eiker kommune - Øvre Eiker kommune \\ Rapport om SAMRUS prosjektet}
\author{Pål Ager-Wick}
%\date{} % Activate to display a given date or no date (if empty),
         % otherwise the current date is printed 


\begin{document}


\newcommand\nbvspace[1][3]{\vspace*{\stretch{#1}}}
% allow some slack to avoid under/overfull boxes
\newcommand\nbstretchyspace{\spaceskip0.5em plus 0.25em minus 0.25em}
% To improve spacing on titlepages
\newcommand{\nbtitlestretch}{\spaceskip0.6em}
\pagestyle{empty}
\begin{center}
\bfseries
\nbvspace[1]
\Huge
{\nbtitlestretch\huge
Prosjektrapport}

\nbvspace[1]
\normalsize

Rus- og psykiatriprosjekt\\
Nedre- og Øvre Eiker kommuner\\
\nbvspace[1]
\small av\\
\Large Pål Ager- Wick\\[0.5em]
\footnotesize Kommuneoverlege\\

\nbvspace[2]

\includegraphics[width=4.5in]{./pix/svirrende.jpg}
\nbvspace[3]
\normalsize

Mjøndalen\\
\large
22.10.2013\\
\nbvspace[1]
\end{center}




\definecolor{f_page}{rgb}{0.6,0.6,0.7}%denne og etterfølgende def bakgrunnsfarge på tittelsiden
\pagecolor{f_page}\afterpage{\nopagecolor}


%%%\setcounter{section}{0}


              \renewcommand{\abstractname}{Oppsummering}

              \begin{abstract}
              \rule{\textwidth}{1pt}\\%horisontal linje
                \paragraph{Skaffe en oversikt\\}
                  Prosjektet hadde som mål å få en oversikt over hvordan kommunene Nedre- og Øvre Eiker samarbeider med spesialisthelsetjenesten i dag. Videre skulle vi se hvor skoen trykket om det er noen områder som burde prioriteres når samhandlingsreformen implementeres også i psykisk helse. Dette oversiktsarbeidet skulle munne ut i å beskrive noen arbeidsmodeller som kunne forbedre og forberede psykiatritjenesten på kommende utfordringer. Følgende punkter er hentet fra prosjektbeskrivelsen\cite{ProPlan-1}.

              \begin{itemize}
                \item Beskrive organisering (ansvar rollefordeling), helhetlig behandlingsmodell og pasientflyt i kommunal psykiatri- og rustjenesten, andrelinjetjenesten, fastleger og NAV
                \item Tiltak for bruk av IP aktivt ovenfor alle pasienter
                \item Utarbeide prosedyre for hvem og hvordan somatisk helse sikres hos pasientgruppen
                \item Lage system for at alle pasienter skal ha arbeid og aktivitetsplan
                \item Definere roller og ansvar slik beskrevet i resultatmålene
                \item Lage et system for systematisk å motta, bearbeide og videreformidle bekymringsmeldinger til rette instans
                \item Tilpasse veilederen for internkontroll til driften i de respektive kommunale tjenester
              \end{itemize}
                \paragraph{Komplekst bilde\\}
                  Det lot seg ikke gjøre å gjennomføre alle hovedtiltakene innenfor prosjektperioden. Det skyldes at jobben med kartlegging og eksisterende samhandling mellom se forskjellige tjenestene var mye mer omfattende enn først antatt. Det finnes i dag få prosedyrer i psykisk helse og etter styringsgruppens mandat valgte vi å fokusere på beskrivelsen av dagens system. Vi ser for oss at dette kan være verdifull informasjon å bygge tjenestene videre på for kommunen og gi andrelinjetjenesten en oversikt over kommunens tilbud i dag. Det er enighet i gruppa om at utviklingsarbeid av tjenestene i dag har en liten plass i arbeidshverdagen og at det er et behov for å videreføre denne typen arbeid. Det kan gjerne følge opp hovedtiltakene som ble gitt denne gruppen.\\
              \newpage
              Gruppen bestod av:\\
              \begin{itemize}
                \item \textbf{Pål Ager- Wick}, prosjektleder og kommuneoverlege i begge kommuner\\
                \item \textbf{Rita Winness}, prosjektsekretær\\
                \item \textbf{Karin Hassellind}, leder psykisk helsetjeneste Nedre Eiker\\
                \item \textbf{Marit Kolbræk}, leder psykiatri boliger Nedre Eiker\\
                \item \textbf{Sveinlaug Barstad}, Leder for koordinerende enhet Øvre Eiker\\
                \item \textbf{Eli Julton}, Leder for psykisk helsetjeneste Øvre Eiker\\
                \item \textbf{Helen Kvalheim}, Drammen DPS avdeling Torsberg(Prosjektleder Samhandlingsteam)\\
                \item \textbf{Sille KvamAnne- Hilde Rese}, Klinisk sosionom, Kongsberg DPS(vara: Anne- Hilde Rese)\\
                \item \textbf{Sissel Halvorsen}, ruskonsulent, NAV Nedre Eiker\\
                \item \textbf{Astrid Rønning}, veileder, NAV Øvre Eiker\\
              \end{itemize}

              \end{abstract}

              \renewcommand{\chaptername}{Del}
              \renewcommand{\contentsname}{Innhold}
              \renewcommand\listfigurename{Illustrasjoner}
              \renewcommand\tablename{Tabell}
		\renewcommand\listtablename{Tabeller}
              \renewcommand{\figurename}{Illustrasjon}

              \tableofcontents

              \chapter{Oversikt og bakgrunn}\label{chap:ov}

                \section{Utfordringer}\label{sec:ov_utf}
                  \paragraph{Sterke og svake sider\\}
                   Det er mye som fungerer godt i dagens system, men det er preget av ineffektive pasientløp, tidvis vanskelig kommunikasjon og tilfeldigheter som avgjør om pasienter får tilbud de skal ha. Det er lite systematikk i bruk av IP og ofte utelates det helt. IP er ikke i alle tilfeller egnet, men at pasienter bør ha en plan og et forløp er det enighet om. \\
                  \paragraph{Mye godhet, store oppgaver\\}
                    Måten psykiatrien er organisert på er et bilde av hva som er godt med det kommunale helsetilbudet: Det er stor omsorg og evne til å løse vanskelige problemer men tidvis forekommer manglende organisering og faglige begrunnede behandlingsløp\label{stor_vilje_til_hjelp}. Dette gjør at enkeltindivider får god oppfølging mens andre ramler utenfor. \\
                  Fra helseforetakets side er stadige endringer med fokus på effektivisering, flytting og endring av ansvarsområder en stor utfordring. Både internt for helseforetaket og for kommunene som skal samarbeide med dem.\\

                \section{Bakgrunn}\label{sec:ov_bakg}
                  \paragraph{Ikke stykkevis, men helt\\}
                    Som i mange andre prosjekter starter vi også her med Stortingsmelding nummer 47, Samhandlingsreformen
                    \cite{Stmld47}. Budskapet til reformen og forarbeidene har vært at ved å bli bedre på samarbeid, ikke jobbe på hver vår tue, kan vi gi bedre pasientbehandling(alibiet) og spare penger(det virkelige målet). \\
                  \paragraph{Store faglige utfordringer\\}
                    Mens de ikke sjelelige sykdommer allerede er 1,5 år i denne reformen, kommer det signaler om at psykiatrien også vil bli inkludert. Det er vel og bra men det er en del nasjonale og lokale utfordringer i disse sykdomsgruppene som foreløpig er uavklart. \\
                   \paragraph{Upopulære planer\\}  
                    Det er ingen enighet om hva som er definert som behandling av psykiatripasienter i kommunen. Det mangler avgjørende kompetanse for å avgrense behandlingsløp. Det synes å være problemer med å skille kurative og kroniske pasientforløp. Det er også motstand mot å bruke IP da dette er sett på som et statisk og i mange tilfeller lite egnet verktøy. Dette tyder på at man verken tror på effekten av en individuell plan eller at den er vanskelig å implementere. Det bør være et sterkt signal fra tjenestene at man ikke evner å innføre lovpålagte virkemidler.\\
                  \paragraph{Engasjerte ansatte og sterk vekst\\}
                    Likevel synes det at mange spennende samarbeidsmodeller vokser fram organisk omkring pasienter på enkeltpersonnivå. Vi er blitt gode til å finne løsninger på tross av systemet, i stedet for på grunn av systemet. Det er også en sterk vekst i etterspørselen av tjenester fra kommunal psykiatri. Tjenestene i begge kommuner rapporterer om at de opererer tett oppunder hva som er maksimalt utnyttelse av tjenestene sine og at dette går ut over nye søkere og tilbudet som forventes i befolkningen.
                  \paragraph{Utfordringer på systemnivå\\}  
                    Det har også de siste 40 årene vært en sterk nedbygging av statlig psykiatriske institusjoner, både nasjonalt og regionalt
                    \cite{SSButtrekk1}. Det er halvert antall heldøgnsplasser siden 1990. Disse pasientene bor nå i kommunene og skal ha tilbud herfra. 

                \begin{figure}[ht]
                  \centering
                  \includegraphics{./pix/heldgnpsykplass}
                  \caption[Oversikt over reduksjon i \textit{heldøgnsplasser}.]
                   {Denne oversikten er basert på et uttrekk fra SSB \textit{heldøgnsplasser}. Det er en sterk reduksjon i disse plassene siden tidlig 90-tall. Det er vanskelig å forstille seg hvordan kommunen alene skal håndtere alle disse uten hjelp av innleggelser.}
                \end{figure}
                  \paragraph{Faglig overbygning\\}
                    Kommunene har ikke ansatt psykiatere og i liten grad ansatt psykologer slik at det meste av arbeidet utføres av sykepleiere med eller uten spesialisering i psykiatri og fastlegene. Men kommunen skal ikke agere sykehus heller, men det synes å være en manglende faglig overbygning. Selv sykepleiere med spesialkompetanse er ikke i stand til å handle ut over sin faglige kompetanse. Det innebærer, ikke endre- eller iverksette ny behandling, ikke iverksette- eller endre medisinering.
                \paragraph{Hvem alle har ansvaret, men hvem har kunnskapen?\\}
                    Dette er en stor utfordring fordi de færreste pasienter har behandlingsopplegg definert av spesialisthelsetjenesten eller fastlegen. Fastlegene kan ha manglende kompetanse på psykiatripasienter og er også avhengige av et tett samarbeid med DPS. 
                    Sykepleierne i psykiatritjenesten havner daglig i situasjoner hvor de må ta avgjørelser som grenser opp mot deres lovpålagte begrensinger i helsepersonellovens §4, noe som i sin tur bryter den kommunale helse og omsorgstjenesteloven
                    \cite{HOTJL-12}.
                    Dette er store nasjonale utfordringer, som forplanter seg til kommunene. Utfordringer som vi ikke er rustet til å møte med dagens organisering. \\

                \chapter{Arbeidsmetode}\label{chap:meto}
                  \section{Møter}
                    \paragraph{Konkret og saklig\\}
                      Møtene har vært strukturert rundt målene som ble angitt i prosjektbeskrivelsen. Det har vært møter omlag en gang i måneden. Der ble problemstillingene fra prosjektbeskrivelsen forsøkt belyst. Det var gruppens uttalte mål å være konkrete og å se på hvilke områder som ville ha størst nytte av bedring av samarbeidet\cite{ProPlan-1}.
                    \paragraph{Fokus på det nåværende\\}
                      Det har vært lagt mest ressurser i kartlegging av den nåværende situasjonen. Det er umistelig viktig å forstå hvordan tjenestene fungerer sammen i dag for å kunne forbedre og endre dem.\\
                    \paragraph{Skreddersøm\\}
                      De første møtene var viet oversikten, men vi valgte også å bruke endel tid på praktiske eksempler. Jo næremere eksempler og vanskelig kasus vi jobbet jo lettere var det å holde engasjementet oppe. Det manglet ikke på praktiske erfaringer og løsninger i enkelttilfeller. 
                  \section{Kurs}
                    \paragraph{Påvirkning utenfra\\}
                      Det var i utgangspunktet ikke intensjonen at vi skulle bruke mye midler på ekskursjoner men Sissel Halvorsen i rusteamet i Nedre Eiker brakte tilbake gode innspill fra deltagelse i kursvirksomhet. Prosjektleder var på "Psykisk Helse 2013" en ansjonal fagkonferanse som gav mange og viktige inspill, og resulterte i mulighetsdiskusjonen omkring bruk av Checkware(les mer på side \pageref{sec:agr_6}).
                    \paragraph{Mer detaljert beskrivelse\\}
                      Resten av denne rapporten er viet resultatet av gruppearbeidet. Det vil tydeliggjøres mer om arbeidsmetodikken gjennom de påfølgende deler, og man velger derfor å ikke gå nærmere inn på arbeidsmetoden her.  

                \chapter{Slik er organiseringen av tilbudet}\label{chap:org}
                  Her følger en beskrivelse av de enkelte møtene, med mål for møtet og hva som ble diskutert. Det ble tidlig viktig å bruke mye tid på diskusjon hva som er status per i dag.\\
                  For eksempel er det ikke lett å lage en konkret modell for implementering av IP som arbeidsverktøy dersom vi ikke er ferdige med å diskutere hvordan arbeidet vårt er organisert.\\

                  \section{Oversikt over tjenestene i kommunen:}
                   
                  \subsection{Øvre Eiker}\label{sec:org_oek}
                      
                      \subsubsection{NAV}
                        \begin{itemize} 
                          \item Har ingen registrering av pasienter med rus- og eller psykiatriske problemer. Det er svært varierende hvor mye oppfølging forskjellige pasienter får. Det er ingen formaliserte samarbeidsrutiner
                        \end{itemize}
                      
                      \subsubsection{Rus- og Psykiatritjenesten}
                      	
                        \begin{itemize}
                    	      \item Jobber mest ambulerenede. Har omlag 300 registrerte brukere eller pasienter. Det er ingen oversikt over hvilke pasientløp som følges. Ingen registrering av pasienter etter  diagnoser. Ingen fastlagte behandlingsløp for grupper eller enekltpasienter. Det meste av arbeidet brukes oppsøkende. \\
                            \item Det tilbys omlag to "Kurs i Depresjonsmestering"(KID) per år. Det er ingen formaliserte samarbeidsavtaler med Nedre Eiker. \\
                            \item Samarbeidet med DPS i Kongsberg beskrives som "nært og godt" og uten stort forbedringspotensial.\\
                            \item Det vanskeligste er den store pågangen av pasienter som også er vanskelige å avslutte. Det kan også være frustrerende med den store mengde akutte hendelser som stjeler tid fra andre oppgaver.\\ 
                            \item Rustjenesten følger opp LAR brukere og andre rusmisbrukere. De tilbyr ADDIS kartlegging og veileder i daglige oppgaver, søker inn til rehabilitering og følger opp kommunens ansvar.\\
                            \item Det foreligger ingen tall på hvor mange henvisinger som kommer eller blir avvist. Det er laget en mal for henvising fra fastlegene. \\
                            \item Tjenesten er organisert med tjenesteleder og har ellers flat struktur. Det er også et dagtilbud hvor pasienter kan møte på et dagsenter for lavterskel aktivitet. \\
                            \item Tjenesten følger også opp pasienter i boliger, samt kjøpte tilbud for pasienter med særlig store hjelpebehov. \\
                        \end{itemize}
                    

                  \subsection{Nedre Eiker}\label{sec:org_nek}
                    \subsubsection{NAV}\label{sec:org_nek_nav}
                      \begin{itemize}
                        \item Jobber tre ruskonsulenter i totalt 2,5 stillinger. Integreringen bidrar til godt samarbeid og integrasjon med NAV-tjenester, men veien til andre kommunale tjenester oppleves som lang noen ganger. 
                      \end{itemize}
                    \subsubsection{Psykiatritjenesten}\label{sec:org_nek_psyk}  
                      \begin{itemize} 
                        \item Jobber også mest ambulerende, men har også pasienter på kontoret. 
                        \item Boliger og dagtilbud er underlagt psykiatritjenesten.
                        \item Ungdomstilbud?
                        \item KID kurs to ganger per år.
                        \item Jobber med å utvikle et eget oppfølgingsteam for særlig behovskrevende brukere. 
                        \item Administrerer oppfølging med urinprøver av metadonbrukere. 
                        \item Oppfølging fra DPS månedlig.
                      \end{itemize}
                  
                      
                  \section{Oversikt over tjenestene i Helseforetakene:}
                    \subsection{Drammen DPS}\label{sec:org_ddps}
                      \subsubsection{Akuttteamet}\label{sec:org_ddps_akutt}
                          \begin{itemize}
                            \item Tilbyr akutt psykisk intervensjon til pasienter med selvmordsrisiko\\
                            \item Kan kommme i hjemmebesøk\\
                            \item Tar i mot henvising fra fastlege og kommunale psykiske helsetjenester\\
                            \item Nylig reorganisert\\
                          \end{itemize} 
                        \subsubsection{Rehabteamet}\label{sec:org_ddps_rehab}
                          \begin{itemize}
                            \item Er ansvarlig for lengre oppføging av pasienter med rehabiliteringspotensiale\\
                            \item Følger opp ambulant pasienter med tyngre psykiske lidelser og samarbeider med kommunen\\
                          \end{itemize}
                        \subsubsection{ARA}\label{sec:org_ddps_ARA}
                          \begin{itemize}
                            \item Ruspoliklinikk, med oppfølging av rus pasienter\\
                            \item LAR behandling, starter og følger opp dette sammen med kommunen\\
                            \item Noen plasser til frivillig avrusning\\
                            \item Anbefaler egnede behandlingssteder og henviser ved kommunale tvangsbehandlingssaker\\
                          \end{itemize}
                        \subsubsection{Poliklinikken}\label{sec:org_ddps_poli}
                          \begin{itemize}
                            \item Oppfølging og behandling av tradisjonelle psykiatriproblemstillinger\\
                            \item Bistår i konferering på dagtid\\
                            \item Koordinerer gruppetilbud også\\
                          \end{itemize}
                        \subsubsection{Lier}\label{sec:org_ddps_lier}
                          \begin{itemize}
                            \item Døgnplasser både på tvang og frivillig\\
                            \item Eneste alternativ ved akutte psykiatriske innleggelser\\
                            \item Bemanner vakttelefonen på kveld- og i helger\\
                          \end{itemize}  
                        \subsubsection{Samhandlingsteamet}\label{sec:org_ddps_samh}
                          \begin{itemize}
                            \item Nyopprettet og har en fleksibel rolle\\
                            \item Følger opp sammensatte pasienter som ellers ikke blir fanget opp av systemet\\
                            \item Er tverrfaglig sammensatt, og har medlemmer fra Lier og Drammen\\
                          \end{itemize}  
                        
                  \subsection{Kongsberg DPS}\label{sec:org_kdps}
                      \subsubsection{Akuttandelingen?}\label{sec:org_kdps_akutt}
                          \begin{itemize}
                            \item Bistår i vurdering av akutt suicidalitet\\
                            \item Hjelper til i prosessen ved videre behov for psykiatrisk behandling\\
                          \end{itemize} 
                        \subsubsection{Rehabteamet}\label{sec:org_kdps_rehabliltering}
                          \begin{itemize}
                            \item Lengre oppfølging av pasienter med psykiatriplager\\
                          \end{itemize}
                        \subsubsection{Ruspoliklinikken}\label{sec:org_kdps_ruspol}
                          \begin{itemize}
                            \item Tilbud til rusmisbrukere med samtaler og tilrettelegging for avrusning\\
                            \item Kartlegging av misbruk\\
                          \end{itemize}
                        \subsubsection{Poliklinikken}\label{sec:org_kdps_poli}
                          \begin{itemize}
                            \item Behandler og følger opp pasienter med psykiatriske lidelser\\
                            \item Koordinerer gruppetilbud også\\
                          \end{itemize}
                        \subsubsection{Døgnposten}\label{sec:org_kdps_dogn}
                          \begin{itemize}
                            \item Planlagte innleggelser for psykiatrisk døgnbehandling\\
                            \item Frivillige og ikke akutte inneleggelser i første rekke\\
                          \end{itemize}  
                        \subsubsection{ACT-teamet}\label{sec:org_kdps_act}
                          \begin{itemize}
                            \item Avvikles i 2013, men har håndtert komplekse pasientgrupper\\
                            \item Se egne inklusjonkriterer\\
                            \item Samarbeid med Øvre Eiker kommune\\
                          \end{itemize}  
                        




              \chapter{Møtene}\label{chap:m_main}
                \section{Møteformat og gjennomføring}\label{sec:m_form}
                  Det ble gjennomført 6 møter i arbeidsgruppen og to møter i styringsgruppen. Ett av møtene i arbeidsgruppen var et heldagsmøte.
                \section{Oppsummering}\label{sec:m_sum}  
                  Med unntak av ett møte ble samtlige gjennomført som planlagt. Prosjektet ble sterkt forsinket på grunn av feil i saksbehandlingen fra Helsedirektoratet og midler ble derfor overført. Prosjektet må ses på som en del av en helhetlig tilpassing til forventede krav fra samhandlingsreformen og et middel for å bedre pasientflyt og -tilbud. Dette i tråd med "Tid, Tillit og tilgjengelighet"~\cite{tid_til_tilgj} og "Mennesler med psykiske lidelser"~\cite{m_psyk_lid}.
                \section{Styringsgruppen}\label{sec:m_stygr}
                  \subsection{\nameref{sec:m_stygr}s første møter}\label{sec:stygr_1}
                 Det er med vilje at det er angitt to "første møter" for styringsgruppen. Det er fordi gjennomføringen av prosjektet stod og falt på delfinansierng fra helsedirektoratet. Dette er svært pressede områder i helsesektoren og man måtte frikjøpe blant annet prosjektssekretær for å gjennomfre prosjektet meningsfullt. \\
                 Det ble gjennomført ett møte 08.02.2012 hvor prosjektdirektivet\cite{ProPlan-1} ble gjennomgått og endret før politisk behandling og gjennomføring. Resultatmål ble fastsatt til:
                  \begin{itemize}
                  \item Det skal være beskrevet en modell for enhetlig jobbing i rus og psykiatritjeneste\\
                  \item Det skal være etablert et system for at hver pasient får en plan for arbeid og aktivitet i løpet av oppfølgingen\\
                  \item IP skal være implementert som arbeidsverktøy\\
                  \item Det skal være beskrevet en modell for pasientflyt og ansvarsplassering\\
                  \item Det skal være etablert et bekymringsmeldingssystem\\
                  \item Sikre en helthetlig utredning som også innbefatter somatisk sykdom.\\
                  \item Veilederen for interkontroll for helse og sosial skal gjennomføres i prosjektets arbeid.\\
                  \end{itemize}

                Av forkjellige grunner, deriblant forsinket saksgang i Helsedirektoratet og endring av organisering og kommunalsjefer i respektive kommuner ble det avholdt et nytt oppstartmøte for styringsgruppen 22.01.2013. Her ble kommunalsjefen i Nedre Eiker, Tor Erik Befring valgt til leder av styringsgruppen og Pål Ager-Wick, kommuneoverlege valgt til leder av arbeidsgruppen. Rita Winness ble ansatt som prosjektsekretær. Det ble holdt et foredrag fra et lignende prosjekt i Vestfold fylke \cite{sporVf}som ble svært godt mottatt. 
                Konklusjonen av møtet var \cite{strgr_mref13-1}Som en tydeliggjøring av arbeidsgruppas mandat:
                  \begin{itemize}
                  \item Finn områder hvor det i dag er stort behov for å utvikle samhandlingsmodeller for pasienter med rusproblemer-, psykiske helseplager eller begge deler.\\
                  \item Informer Styringsgruppen om hvilke områder som er funnet og velg 1-2 områder som skal være tema for arbeidsgruppen.\\
                  \item Lag en plan for å registrere måloppnåelse med utgangspunkt i Helsedirektoratets kriterer for tildeling av midlene.\\
                  \end{itemize}


                  \subsection{\nameref{sec:m_stygr}s andre møte}\label{sec:stygr_2}
                    Etter omlag et halvt år med jobbeing i arbeidsgruppen møttes styringsgruppen igjen \cite{strgr_mref13-2} for en oppdatering av hvordan arbeid hadde gått og å komme med innspill til innhold i de siste møtene. Innspillene til denne rapporten var:
                    \begin{itemize}
                    \item De ulike pasientforløp gir et mangeforgrenet og uoversiktlig behandlingstilbud. Kanskje vi skal ha flere team for ulike diagnosegrupper?\\
                    \item Vi bør se på hvilke tilbud vi skal tilby og hvilke vi må tilby\\
                    \item Det trengs kartlegging av hvor mange av brukerne som forblir i systemet\\
                    \item Flere bør benytte de statlige tilbudene gjennom nav sitt kvalifiseringsprogram\\
                    \item Bruke “tjenestedesign” for å sette pasienten i sentrum?\\
                    \item Rapporten bør inneholde forslag på interkommunalt samarbeid og hvordan man kan iverksette forslagene til sømløs pasientforløp. Utprøving og implementering blir et senere prosjekt.\\
                    \end{itemize}
                    Det var lite støtte i gruppa for å utprøve softwareløsninger i bare en kommune selv som en pilot(se side \pageref{cw_pilot}, \ref{cw_pilot})
                \section{Arbeidsgruppen}\label{sec:m_agr}
                  \subsection{\nameref{sec:m_agr}s første møte}\label{sec:agr_1}
                    Arbeidsgruppens første møte\cite{arbgr_mref-1} handlet om presentasjon av de forskjellige tjenestene i alle linjene. Mandatet ble gjennomgått og det var en åpen diskusjon om hvem som har størst behov for å koordinere dagens tjenestetilbud. Hvordan vi er organisert og jobber med dette i dag og hva som er de største utfordringene. \\

                    Den største utfordringen var å orientere seg i landskapet for psykiske helsetjenster i i Eikerkommunen og hvordan de samhandler med sykehusene. Sykehustilbudene er heller ikke enhetlig organsiert men tilbyr flere lignende tilbud i hver sin organiseringsform. Som relativt utenforstående var det ikke lett å få en oversikt. Det er gitt tilbakemeldinger fra blant annet fastlegene at tjenestene ikke er lette å orientere seg i. \\ 
                    \\
%Her begynner grafikken

                    \begin{figure}[ht]
                      \centering
                      \includegraphics{./pix/kaotiskvei.jpg}%placeholder
                      \caption[Framstilling av tjenestetilbudene]%\textit{tjenestetilbudene}.]
                      {Det kan være et forvirrende landskap å orientere seg i. Ikke minst for pasienten.\cite{shutter}}
                    \end{figure}    
%Her slutter grafikken                  
                  \subsection{\nameref{sec:m_agr}s andre møte}\label{sec:agr_2}
                    Etter at alle hadde blitt litt bedre kjent med hvordan man er organisert og hva slags tilbud vi har handlet dette møtet om praktiske utfordringer knyttet til pasienteksempler. Prosjektlederen hadde forbredt fire kasus som ble diskutert i gruppen. Som beskrevet på side \pageref{stor_vilje_til_hjelp} er det stor vilje til å hjelpe og diksusjonen omkring pasientene bar preg av dette\cite{arbgr_mref-2}. Det ble holdt en innledning som var tilpasset av Simon Sinek\cite{sinek09}. Det er viktig at man holder fokus på hvorfor vi gjør det vi gjør og bygger opp tjenesten derfra. \\
                    Ut fra pasientene som ble diskutert kom vi fram til noen nøkkelemner vi skulle jobbe videre med:\\
                      \begin{itemize}
                        \item Påstand: Man kan lage kriterier for pasienter med ekstra behov\\
                        \item Hva er definisjonen av behandling i kommunen?\\
                        \item Hvilke kriterier kan og bør kommunen ha for å gi behandling?\\
                        \item Alle som er utfordrende for systemet bør ha en kriseplan\\
                      \end{itemize}
                    En av hovedutfordringene vi står igjen med er at det er vanskelig å identifisere de som trenger skreddersydd behandling. Det går ofte flere år fordi vi ikke er klare over at alle må jobbe koordinert omkring pasienten eller brukeren for å kunne være effektive. Mens tiden går blir pasienten ofte dårligere og det blir tiltagende vanskelig å komme i god behandlingsposisjon.   

                  \subsection{\nameref{sec:m_agr}s tredje møte}\label{sec:agr_3}
                    Møtet tok utgangspunkt i punktene som ble angitt fra forrige møte\cite{arbgr_mref-2}. Det ble derfor gjort et forsøk på å finne kriterier for å identifisere pasientene vi snakket om i avsnittet\ref{sec:agr_2}. Målet var å finne kriterer som kunne identifisere personer med behov uvhengig av hvilken tjeneste som var i kontakt med dem. \\
                    \subsubsection{Forslag til kriterer:}
                      \begin{itemize}
                        \item Er det kommet en bekymringsmelding?\\
                        \item Er pasienten over 16 år?\\
                        \item Har personen jobb eller aktivitet?\\
                        \item Har noen av følgende allerde vært involvert?
                          \begin{itemize}
                            \item PPT\\
                            \item Politi\\
                            \item Utekontakt\\
                            \item Skolehelsetjenesten\\
                            \item Barnevern\\
                            \item Helsestasjon\\
                            \item Familiesenteret\\
                            \item Fengsel\\
                          \end{itemize}
                        \item Avbrutt skolegang?\\
                        \item Rus eller psykiatrihistorikk?\\
                        \item Møter ikke til avtale\\
                      \end{itemize}
                    Dette var ment som en sjekkliste som alle kunne bruke. Ved ja på 5 eller flere spørsmål kan pasienten eventuelt henvises til et team for de som trenger særlig oppfølging, eller man oppnevner en koordinator for pasienten. Arbeidsgruppen er klar over at alle pasienter eller brukere med behov for langvarig oppfølging vil ha behov for en IP, men i praksis er det lurt å fokusere innsatsen hos de som har størst behov. Vi mener at slike kriterer kan bidra til å fokusere innsatsen.\\
                    \subsubsection{Prioritering i kommunen}
                    Diskusjonen gikk også rundt problemet kommuenen har med å avslutte pasienter, samt å prioritere pasienter inn til et tilbud. Fra Øvre Eiker hersker det usikkerhet om pasienter kan avvises, eller ikke gis et tilbud. I gruppen blir det konkludert med at når det foreligger faglige grunner er dette ikke noe problem. Det bør derimot være klart hvem kommunen skal og kan hjelpe.

                  \subsection{\nameref{sec:m_agr}s fjerde møte}\label{sec:agr_4}
                    Her fikk vi presentert modellen som er foreslått inn i Drammensregionen for Lier og Drammen i samarbeid med Drammen DPS. Helen Kvalheim presenterte modellen som skal føre til bedre bruk av ressursene i DPS og kommunene for de mest utfordrende brukerne. Noen nøkkelmomenter er at psykiater og psykolog sammen med ansatte fra kommunen i deltidsstillinger skal sørge for en helhetlig oppfølging. 
                    \subsubsection{Samhandlingsteamet i Drammen:}
                     \begin{itemize}
                      \item Tett på, langsiktig oppfølging\\
                      \item Skulder til skulder\\
                      \item Kort vei til veiledning\\
                      \item Få tak i "såpestykkepasienten"\\
                     \end{itemize}
                    \subsubsection{Fastlegenes rolle}
                      Fastlegene spiller alltid en viktig rolle i alle pasientenes oppfølging, men kan noen ganger være et utfordring å samarbeide med melder flere av kursdeltagerne. Det gis en innføring i legens takstsystem og forslag til å utforme brev slik at fastlegene lettere kan benytte takstsystemet. Det er også en kommunikasjonsutfordring, fordi fastlegene kan være vanskelig å få tak i. Her vil sannsynligvis elektroniske meldinger lette noe av byrden, men bare dersom det ikke blir mye unødvendig kommunikasjon.

                  \subsection{\nameref{sec:m_agr}s femte møte}\label{sec:agr_5}
                    Møtet var svært variert med mange tema, se~\cite{arbgr_mref-5} for utfyllende informasjon. Først kom gruppen med en utalelse i høringen om en ny spesialitet for rusmedisin skal implementeres i legeutdanningen.\\
                    \subsubsection{Forbedre ut fra dagens situasjon}
                      Tiltakene som ble diskutert her hadde vært oppe i de tidligere gruppene, men igjen forsøkte vi å sammenstille de viktigste slik at man får et håndfast utgangspuntk for det videre arbeidet. Å avklare forventningene er viktig og både kommune og sykehus føler på fastlegenes varierende innsats. Særlig hadde gruppens deltagere opplevd:\\
                        \begin{itemize}
                          \item Fraværende på samabeidsmøter om de sykeste\\
                          \item Mangelfull somatisk oppfølging\\
                          \item En nøkkelperson til å få løst problemer\\
                        \end{itemize}
                      IP, boligsosialt arbeid og dialog som er satt i system var andre punkter som det var gjort forkjellige erfaringer med. 
                    \subsubsection{Helhetlig pasientforløp}
                     Det feier en vind av helhetlig pasientforløp gjennom landet og Eikerkommunene er helt i front i arbeidet med dette. Utvikling av kommunal helsetjeneste har ført til en industriproduksjon av helsetjenester og nå kommer hele forløpet i fokus. Dette er heller ikke psykisk helse og rus foruten og det var en kort diskusjon om hvordan dette vil påvirke behandlingen i dag. 
                    \subsubsection{Velg én ting...}
                      Gruppedetagerne fikk på slutten av møtet sprøsmålet: - Hvis du kunne velge én ting som du kunne forandre på i din tjenestehverdag, hva ville det være?\\
                      Her følger noen av svarene:
                        \begin{itemize}
                          \item Ville gjerne satt brukeren mer i sentrum!\\
                          \item Involvert fastlegene mer, de kikker alltid på klokka.\\
                          \item Flere lavterskeltilbud å tilby\\
                          \item Én dør inn i rusoppfølgingen i andrelinjetjenesten, opplever at man ikke vil ha pasienter henvist internt\\
                          \item Mer fleksibilitet i oppfølgingen når pasienter flytter mellom kommunene, så man ikke begynner på bar bakke.\\
                          \item Tettere samarbeid med ruskonsulentene\\
                        \end{itemize}

                  \subsection{\nameref{sec:m_agr}s heldagsmøte}\label{sec:agr_6}
                    Det avsluttende møtet ble lagt til Portåsen i Nedre Eiker for å kunne gi gruppen mer sammenhengende tid til å:
                      \begin{enumerate}
                        \item Komme med anbefalinger om en helhetlig behandlingsmodell.\\
                        \item Komme med forslag til konkrete prosjekter som man kan jobbe videre med etter dette for-prosjektet er avsluttet\\
                      \end{enumerate}
                    Møtet oppsummerte i stor grad de foregående møter og konklusjonenen og anbefalingene er å finne i de følgende kapitler og oppsummeringen. Her ble også IT- verktøyet \href{http://checkware.com/}{Checkware} presentert. Dette programmet var blitt presentert i Nedre Eiker og vil kunne gi kvalitative behandlingsdata om pasienter med oppfølging fra psykisk helsetjeneste. På grunn av at dagens journalprogram "Profil" levert av Visma vil ikke prioritere en tilpasning for rapporter. Dette til tross for at andre system kun har brukt ca en arbeidsdag på dette. Fordelene ville innebære at standardiserte psykometriske tester som brukes av DPS ville kunne integreres i journalen både på sykehuset og i kommunen. \label{cw_pilot}Gjennomføringen av et pilotprosjekt er foreløpig strandet da styringsgruppen ikke ønsket å følge prosjektet uten at det ble gjennomført i begge kommuner.


              \chapter{Diskusjon}\label{chap:disk}
                \section{Største utfordringer for oss per i dag} \label{chap:disk_utf}
                  I rus- og psykiatrien har det foregått et forflytning i av behandlingsansvaret fra stort sett innleggelser til behandling og oppfølging nær pasienten. Kommunen er tiltenkt et stadig større ansvar for oppfølgingen her og vil få stadig større del av oppfølgingsansvaret. Et stort problem er at det ikke er definert hvem som skal følge opp, eller lede behandlingen.
                  \paragraph{Tilfeldig skreddersøm\\} 
                    Det virker som den manglende ansvarsplasseringen fører til at ansvartet pulveriseres. Først når alle parter møtes i særlig vanskelige saker kommer man til enighet om hva som må gjøres. Det er ofte tilfeldigheter som fører alvorlige syke pasienter inn i tverrfaglig og god god helhetlig oppfølging. Arbeidshveredagen er preget av mye tid på brannslukking og lite tid på lengre behandlingsforløp.\\
                  \paragraph{Frustrasjon\\}  
                    Det er også språklige utfordringer mellom andre- og førstelinje. Det er ikke alltid klart for førstelinje hvorfor andrelinje- gjør og tenker på bestemte måter omkring pasienter som skrives ut eller ikke legges inn. Avstandene er dels store og dels små. Men kommunene er ofte frustrerte over å ikke bli hørt i spesialisthelsetjenesten. Dette er alvorlig da spesialistene er en del av den faglige overbygningen.\\
                  \paragraph{Sterk vekst\\}  
                    Det er en sterk økning i søknadene om nye tjenester. Det er kun Nedre Eiker som kan levere tall på dette, men Øvre Eiker rapporterer også en sterk opplevd økning i tilstrømningen.\\
                  \paragraph{Ikke Praktisk\\}  
                    Det er mye snakk om IP(individuell plan) som verktøy omkring pasienter. Dette er ikke ansett som et godt verktøy å jobbe med universelt. Det oppleves også som tungvint og lite meningsfullt når pasientene tidvis har ennå større utfordringer.

                \section{Forslag til løsninger}\label{chap:disk_losn}
                  Enkle løsninger er det ikke kommet forslag om, men det er prosjektgruppens mening at et langvarig og målrettet arbeid er nødvendig for å håndtere dette i fellesskap. Les mer om dette i "\nameref{chap:vvidere}".
              \chapter{Veien videre}\label{chap:vvidere}
                \section{Forslag til konkrete prosjekter}\label{chap:vvidere_konkpr}
                  \subsubsection{Checkware} 
                    Det trondheimsbaserte firmaet \href{http://checkware.com/}{Checkware} leverer en helsenett-basert samling av psykometriske tester som kan integreres i de fleste journalsystemer. Psykometriske tester er tester som forsøker å måle psykiatriske symptomer gir et kvalitativt mål på behandlingen som leveres. Systemet er utviklet i samarbeid med Helse midtnorge og gir et innblikk om behandlingen vi gir faktisk virker. Testene er de samme som sykehuset bruker. Styringsgruppen ønsket ikke å starte en pilot i kun en kommune slik det ble foreslått.
                  \subsubsection{Samhandlingsteam Drammen-Lier og DPS}
                    Samhandlingsteamet er startet i oktober 2013. Det er delvis finansiert av Drammen DPS og omliggende kommuner. Det skal fylle en ACT- lignende rolle men ikke være bundet av den nokså rigide ACT-modellen. Drammen DPS har store forventinger knyttet til prosjektet.
                  \subsubsection{Nytt prosjekt i regi av Kongsberg DPS, prosjektleder Sille Kvam}
                    Rett før rapporten ferdigstilles skal Sille Kvam som leder se på ytterligere problemstillinger i samhandlingsaksen i regi av Kongsberg DPS. 
                  \subsubsection{Boligsosialt arbeid i Nedre- og Øvre Eiker}
                    Det pågår prosjekt sammen med omliggende kommuner om boligsosialt arbeid. Dette er nært knyttet til arbeid med rus- og psykiatri i kommunen.
                  \subsubsection{Systemeatisk arbeid med midler fra helsedirektoratet}
                    Det er ansatt en person til å ha overblikk med midler fra helsedirektoratet i Nedre Eiker. I Øvre Eiker er arbeidet med dette ikke systematisert, men det er flere som følger med på tilskuddsportalen.
                  \subsubsection{Psykolog i Kommunen(Nedre Eiker)}
                    Søknad er sendt og familiesenteret er tiltenkt en psykologstilling som fases inn med midler fra Helsedirektoratet.
                \section{Prosjektgruppens anbefalinger}\label{chap:vvidere_anbef}
                  \begin{itemize}
                    \item Det må jobbes videre med å samordne kommunale og sykehustjenestene\\
                    \item Det bør brukes samme verktøy i kartleggingen av pasienter i kommune og spesialisthelsetjeneste\\
                    \item Det bør gjennomføres en pilot med Checkware\\
                    \item Alle psykisk helse prosjekter bør ha et samordningsnettverk, alle prosjektene tar veldig mye tidog overlapper tidvis.\\
                  \end{itemize}
                \section{Søknad om midler til videre prosjekter}\label{chap:vvidere_soknmid}
                  Det er flere midler kommunene og sykehuset kan søke på. Enten i fellesskap, eller hver for seg. Nedenfor følger en liste over tema slik de er angitt fra helsedirektoratet. Det er varierende datoer og søknadsfrister som dels endres, derfor blir det litt generelt. Det anbefales at en person i hver avdeling har ansvaret for å sjekke helsedirektoratets tilskuddsider for eksempel hver 14. dag slik at man har oversikten over hva som finnes.\\
                  \subsection{Midler til psykolog i kommunene}\label{sec:vv_mid_psyk}
                    Nedre Eiker har per i dag sendt søknad om psykologstilling til kommunen, først og fremst for å jobbe tilknyttet familiesenteret. Øvre Eiker har allerede to psykologer tilknyttet barnevern og helsestasjon. Det er forsatt mangle på psykologspesialister og ingen planer om tilsetting av psykologer i tilknytting til psykisk helseteam i noen av kommunene.
                  \subsection{Midler til iverksetting av ROP tiltak}\label{sec:vv_mid_iverksetting_ROP}
                    Det finnes midler på dette området fra helsedirektoratet i dag, men prosjektet er ikke kjent med at det foregår arbeid- eller er søkt på slike midler.
                  \subsection{Midler til oppfølging av ROP tiltak}\label{sec:vv_mid_oppf_ROP}
                    Heller ikke på dette punktet er det i dag søkt om midler såvidt prosjektet er bekjent. 


                % Her begynner kildehenvisningene
              \renewcommand{\bibname}{Kilder:}
              \begin{thebibliography}{99}

                \bibitem{Stmld47}
                  Helse- og Omsorgsdepartementet,
                  \emph{Stortingsmelding 47, 06/2009, Samhandlindlingsreformen}.
                  Hansen, Bjarne Haakon(Minister)

                \bibitem{SSButtrekk1}
                  Statistisk Sentralbyrå, uttrek basert på online database,
                  \emph{SSB 2013, aggregerte data}.

                \bibitem{HOTJL-12}
                  Helse- og Omsorgsdepartementet,
                  \emph{Helse- og Omsorgstjenesteloven, 2012}.
                  Regjeringen Stoltenberg II

                \bibitem{ProPlan-1}
                  Prosjektbeskrivelse,
                  \emph{Helse- og Omsorgsseksjonene i Øvre- og Nedre Eiker}.
                  Styringsgruppen: Tor Erik Befring (\emph{Leder}), Lisbeth Nymo m. fl.

                  \bibitem{sporVf}
                  Spor Samhandling psykiatri og rus,
                  \begin{itemize}
                  \item \emph{Deltakere Kommunene: Nøtterøy, Tønsberg, Stokke, Sandefjord og Larvik\\
                  \item NAV Vestfold\\
                  \item Psykiatrien i Vestfold HF}\\
                  \end{itemize}
                  Vidar Bjørn, prosjektleder \href{mailto:vibjoe@siv.no}{}\\

                  \bibitem{strgr_mref13-1}
                  Referat fra møte i styringsgruppen samhandlingsprosjekt\\ 
                  SamRus - Øvre og Nedre Eiker\\
                  Tid: Tirsdag 22. jan. 2013,  kl. 12.00 – 15.30\\
                  Sted: Portåsen, Nedre Eiker kommune\\

                   \bibitem{strgr_mref13-2}
                  Referat fra møte i styringsgruppen samhandlingsprosjekt\\ 
                  SamRus - Øvre og Nedre Eiker\\
                  Tid: Tirsdag 11. juni. 2013,  kl. 10.00 – 12.00\\
                  Sted: Formansskapssalen Øvre Eiker kommune\\

                  \bibitem{arbgr_mref-1}
                  Referat fra møte i arbeidsgruppen samhandlingsprosjekt\\ 
                  SamRus - Øvre og Nedre Eiker\\
                  Tid: Fredag 15. februar. 2013,  kl. 12.00 – 14.00\\
                  Sted: Formansskapssalen Øvre Eiker kommune\\

                  \bibitem{sinek09}
                  Sinek, S. (2009). Hvordan store ledere inspirerer til handling. TEDxPuget Sound. TEDx, Ted.com.\\

                  \bibitem{arbgr_mref-2}
                  Referat fra møte i arbeidsgruppen samhandlingsprosjekt\\ 
                  SamRus - Øvre og Nedre Eiker\\
                  Tid: Onsdag 13. mars. 2013,  kl. 09.00 – 11.00\\
                  Sted: Formansskapssalen Øvre Eiker kommune\\

                  \bibitem{arbgr_mref-3}
                  Referat fra møte i arbeidsgruppen samhandlingsprosjekt\\ 
                  SamRus - Øvre og Nedre Eiker\\
                  Tid: Mandag 13. April. 2013,  kl. 09.00 – 11.00\\
                  Sted: Formansskapssalen Øvre Eiker kommune\\

                  \bibitem{arbgr_mref-4}
                  Referat fra møte i arbeidsgruppen samhandlingsprosjekt\\ 
                  SamRus - Øvre og Nedre Eiker\\
                  Tid: Onsdag 29. April 2013,  kl. 09.30 – 11.30\\
                  Sted: Formansskapssalen Øvre Eiker kommune\\

                   \bibitem{arbgr_mref-5}
                  Referat fra møte i arbeidsgruppen samhandlingsprosjekt\\ 
                  SamRus - Øvre og Nedre Eiker\\
                  Tid: Mandag 13. Mai 2013,  kl. 10.00 – 12.00\\
                  Sted: Formansskapssalen Øvre Eiker kommune\\

                   \bibitem{arbgr_mref-6}
                  Referat fra møte i arbeidsgruppen samhandlingsprosjekt\\ 
                  SamRus - Øvre og Nedre Eiker\\
                  Tid: Fredag 14. Juni 2013,  kl. 10.00 – 14.00\\
                  Sted: Portåsen, Nedre Eiker kommune\\

                  \bibitem{m_psyk_lid}Mennesker med alvorlige psykiske lidelser og behov for særlig tilrettelagte tilbud Vurdering av omfang og behov, samt forslag til tiltak\\
                  IS-1554\\
                  Utgitt av: Helsedirektoratet\\
                  Kontakt: Divisjon psykisk helse og rus\\
                  Postadresse: Pb. 7000 St Olavs plass, 0130 Oslo\\
                  Besøksadresse: Universitetsgata 2, Oslo\\

                  \bibitem{tid_til_tilgj}
                  Heftets tittel: Tillit, tid, tilgjengelighet\\
                  Tett individuell oppfølging av mennesker\\
                  med behov for sammensatte tjenester\\
                  Utgitt: 10/2011\\
                  Publikasjonsnummer: IS-1918\\
                  Utgitt av: Helsedirektoratet\\
                  Avdeling: Psykisk helse og rus\\
                  Postadresse: Pb. 7000 St. Olavs plass, 0130 Oslo\\
                  Besøksadresse: Universitetsgata 2, Oslo\\

                  \bibitem{God_rusf}God rusforebyGGinG\\
                  Utgitt av Helsedirektoratet\\
                  Universitetsgata 2\\
                  Postboks 7000 St. Olavs plass\\
                  0130 Oslo\\
                  Tlf: 810 20 050\\

                  \bibitem{shutter}Bildene er (c) Shutterstock, og er gjengitt med tillatelse\\
  




              \end{thebibliography}

              \listoffigures

              \end{document}
